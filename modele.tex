\documentclass{article}[a4paper]

\usepackage[top=2cm,bottom=2cm,left=2cm,right=2cm]{geometry}
\usepackage[T1]{fontenc}
\usepackage[french]{babel}

\usepackage{graphics}
\graphicspath{{images/} {images/util/}}

%Mettre le nom de la matière du TP du jour.
\newcommand{\subject}{Matière enseignée à l'école} 

%Mettre ici le nom du binome qui écrit le rapport.
\newcommand{\binomeUn}{Nom du binôme de TP} 
%Mettre ici l'autre nom qui bosse. 
\newcommand{\binomeDeux}{Nom de l'autre binôme} 
%Mettre ici le numéro de TP
\newcommand{\numTP}{\#}

\title{Compte rendu de travaux pratiques (\subject)\newline TP\numTP}
\author{\binomeUn \ \& \binomeDeux}







%___________________________________________________________________________

\begin{document}
	\maketitle
	\begin{center}
		\includegraphics{"Logo ENSEA"}
	\end{center}
	\tableofcontents
	
	%Décommenter si nécessaire
	%\section*{Introduction}
	
	%vvv Le TP commence ici vvv
	\section{}
	
	%Décommenter si nécessaire
	\section{Conclusion}
	
	%Décommenter si nécessaire
	%\section{Annnexe}

\end{document}







%___________________________________________________________________________
%________AIDE-MEMOIRE_______________________________________________________
%___________________________________________________________________________

%Saut de ligne
%\\

%Grande partie de TP :
%\section{}
%Petite partie de TP :

%Pour faire une remarque : 
%\paragraph{Remarque}

%Note de bas de page : 
%\footnote{Remarque en note de pied de page}

%Pour écrire en mathématique (plus de commandes sur Internet) :
%$<formule>$
%
%\begin{math}
%	<formule sur une nouvelle ligne non numérotée>
%\end{math}
%
%
%\begin{equation}
%	<formule sur une nouvelle ligne non numérotée>
%\end{equation}
%
%
%Faire une figure "flottante"
%\begin{figure}
%	\centering
%	\includegraphics[width=.5\linewidth]{image}
%	\caption{légende}
%	\label{fig:img}
%\end{figure}
%
%Référencer la figure : 
%"voir la figure \ref{fig:img}"
%
%





%En cas de problème, supprimer les fichiers aux/log.
%Pour en apprendre plus voir également :
%
%	LATEX2e: An unofficial reference manual
%
%l'ouvrage qui m'a aidé à apprendre le latex.
%(Utile pour les tables et tabular par exemple)

%___________________________________________________________________________
%____FIN_AIDE-MEMOIRE_______________________________________________________
%___________________________________________________________________________

